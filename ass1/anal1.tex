\documentclass[11pt]{article}
\usepackage[utf8]{inputenc}
\usepackage[letterpaper,top=0cm, margin=0.85in]{geometry}

\usepackage{textcmds} %more symbols
\usepackage{fontspec} %more fonts


%for math
\usepackage{amsmath, amstext, amssymb, amsfonts} %standard
\usepackage{youngtab} % makes squares for math diagrams
\usepackage{microtype} %% <-- added
%-----------------------------------------------------------           

%\usepackage{sectsty}
%for lists and numbers
\usepackage{enumitem}
%-----------------------------------------------------------

% Doc setting
\usepackage[english]{babel} % Replace `english' with e.g. `spanish' to change the document language
\usepackage{setspace} %to set spacing bw words and lines
\usepackage{changepage}
% \setlength\parindent{0pt}

%footer
\usepackage{fancyhdr}
\usepackage{lastpage}

% \fancyhf{0pt} % sets both header and footer to nothing
% \renewcommand{\footrulewidth}{0pt}
% \renewcommand{\headrulewidth}{0pt} %remove headerline

% \fancyfoot[RE,RO]{\thepage}
\fancyfoot[C]{MATH 201 | \emph{Shubhro Gupta}}
\renewcommand{\headrulewidth}{0pt}% Default \headrulewidth is 0.4pt
\renewcommand{\footrulewidth}{0.4pt}% Default \footrulewidth is 0pt
\fancyfoot[L]{Assignment 1}
\fancyfoot[R]{\thepage}
\pagestyle{fancy}
%-----------------------------------------------------------

%for pictures and graphs
\usepackage{graphicx} %add image
\usepackage{adjustbox}

\usepackage{pgfplots} %for graphing plotting
\pgfplotsset{compat=1.18, width=10cm}
%-----------------------------------------------------------

%for code
\usepackage{verbatim}
\usepackage{listings}
\usepackage{fancyvrb} %for coding blocks
%\usepackage{algorithm}
%\usepackage{algpseudocode} %for pseudocode
%\usepackage{algorithm, algpseudocode}

%\usepackage{lstfiracode} %firacode
\usepackage[framemethod=tikz]{mdframed} %adding background to lstlisting
\usepackage[ruled,vlined,boxed]{algorithm2e} %for pseudocode lines



%for colors and links
\usepackage[colorlinks = true,
            linkcolor = blue,
            urlcolor  = blue,
            citecolor = blue,
            anchorcolor = blue]{hyperref}
\usepackage[many]{tcolorbox}  % for colored boxes
\usepackage{color} % to get colors
\usepackage{xcolor} %more colors options and flexibility
\usepackage{transparent}


%-----------------------------------------------------------------------------
%custom commands

%code
\newcommand{\problem
}[2]{
\begin{mdframed}
    \textsc{question} \textbf{#1} \hfill #2
\end{mdframed}
}
\newcommand{\codecap}[2]{{\vspace{4pt}{\emph{#1}}} \hfill \href{#2}{Link to the code\ }\vspace{25pt}}
\newcommand{\code}[1]{{\texttt{#1}}}

%math
\newcommand{\bigo}[1]{$O(#1)$ }
\newcommand{\thetan}[1]{$\theta(#1)$}
% \newcommand{\vector}[1]{$\overrightarrow{#1}$}}o
\newcommand{\contradiction}{%
\begin{tikzpicture}[rotate=45,x=0.5ex,y=0.5ex]
\draw[line width=.2ex] (0,2) -- (3,2) (0,1) -- (3,1) (1,3) -- (1,0) (2,3) -- (2,0);
\end{tikzpicture}
}

\newcommand{\vecset}[2]{\{ {#1}_1, {#1}_2, {#1}_3,  \dots,  {#1}_{#2}\}}

%display
\newcommand{\link}[3][blue]{\href{#2}{\color{#1}{#3}}}%
\newcommand{\inlink}[1]{\underline{\emph{\link[black]{#1}{#1}}}}


%header
\newcommand{\heading}[5]{
\hrule ~\\~\\
\begin{large}
\noindent\emph{#1}\smallskip ~\\
Professor #3 \hfill Week #2 \smallskip ~\\
\textbf{Shubhro Gupta} \hfill Due #4 ~\\
\end{large} \medskip ~\\
{\emph{Collaborators: #5}}~\\
\hrule
\vspace{50pt}
~\\
}

% \newcommand\dunderline[3][-1pt]{{%
%   \sbox0{#3}%
%   \ooalign{\copy0\cr\rule[\dimexpr#1-#2\relax]{\wd0}{#2}}}}


%-----------------------------------------------------------------------------
%title
\usepackage{algpseudocode}
\begin{document}

\heading{Real Analysis}{1}{Rishi Vyas}{Tuesday, July 30, 2024}{Sahithim Akella, Khushi Tyagi }

\problem{A}{4 Points}
\textbf{Definition.} \emph{Let $x \in \mathbb{R}$. If $n \in \mathbb{N}$, we define $x^n:=\overbrace{x \ldots x}^{n \text { times }}$. By convention, for $n=0$ we interpret this as defining $x^0:=1$.}~\\
\\
\textbf{To Prove.}  Let $x, y, z \in \mathbb{R}$.

\begin{enumerate}
	\item If 0' is an element of $\mathbb{R}$ such that $0'+x=x$ for all $x \in \mathbb{R}$, then $0'=0$.
	      \begin{align*}
		      0' + x        & = x                                             \\
		      0' + x + (-x) & = x + (-x)                                      \\
		      0' + 0        & = 0        \qquad \text{(Additive Inverse (4))} \\
		      0'            & = 0                                             \\
		      \square
	      \end{align*}
	\item If 1' is an element of $\mathbb{R}$ such that $1' \cdot x=x$ for all $x \in \mathbb{R}$, then $1'=1$.
	      \begin{align*}
		      1' \cdot x              & = x                                                   \\
		      1' \cdot x \cdot x^{-1} & = x \cdot x^{-1}                                      \\
		      1' \cdot 1              & = 1        \qquad \text{(Multiplicative Inverse (8))} \\
		      1'                      & = 1                                                   \\
		      \square
	      \end{align*}
	\item $x \neq 0$ and $x y=x z$. Prove that $y=z$. Deduce that if $x y=1$ then $y=x^{-1}$.
	      \begin{align*}
		      x y                  & = x z                  \\
		      ((x)^{-1} \cdot x) y & = ((x)^{-1} \cdot x) z \\
		      y                    & = z                    \\
		      \square
	      \end{align*}
	      \begin{align*}
		      x y                  & = 1                \\
		      ((x)^{-1} \cdot x) y & = (x)^{-1} \cdot 1 \\
		      \square
	      \end{align*}
	\item $-0=0$
	      \begin{align*}
		      0           & = 0           \\
		      0 \times -1 & = 0 \times -1 \\
		      0 = - 0                     \\
	      \end{align*}
	\item If $x \neq 0$, then $x^{-1} \neq 0$ and $\left(x^{-1}\right)^{-1}=x$.
	      \begin{align*}
		      \text{Suppose }x^{-1} & = 0                                                    \\
		      x \cdot x^{-1}        & = x \cdot 0                                            \\
		      1                     & = 0         \qquad \text{(Multiplicative inverse (8))}
	      \end{align*}
	      $1 \neq 0$ so $x^{-1} \neq 0$.
	\item $(-x) \times(-y)=x y$.
	      \begin{align*}
		      (-x) \times(-y)  & = (-x)(-y) + 0y                                          \\
		                       & = (-x)(-y) + (x + (-x))y                                 \\
		                       & = (-x)(-y) + xy + (-x)y \qquad \text{(Distributive Law)} \\
		                       & = ((-x)(-y) + (-x)y) + xy                                \\
		                       & = (-x)(-y + y) + xy                                      \\
		                       & = (-x)(0) + xy                                           \\
		                       & = 0 + xy                                                 \\
		      (-x) \times (-y) & = xy
	      \end{align*}
	\item If $x \neq 0$ and $n \in \mathbb{N}$, then $(-x)^{-1}=-\left(x^{-1}\right)$ and $\left(x^{-1}\right)^n=\left(x^n\right)^{-1}$.
	\item If $x \neq 0$ and $y \neq 0$, then $x y \neq 0$.
	      \begin{align*}
		      \text{Suppose } x y    & = 0             \\
		      x \cdot x^{-1} \cdot y & = 0\cdot x^{-1} \\
		      1 \cdot y              & = 0\cdot x^{-1} \\
		      y                      & = 0
		      \\
		      \contradiction
	      \end{align*}


\end{enumerate}

\newpage
\problem{B}{3 Points}
Let $x, y, z \in \mathbb{R}$.
\begin{enumerate}
	\item If $x<y$ and $z>0$, then $x z<y z$.
	      \\
	      \textbf{Proof.}\\
	      If $x < y$, then $y - x > 0$. Since $z > 0$, we can multiply both sides of the inequality by $z$ to get $z(y - x) > 0$. This implies $z y - z x > 0$, which is equivalent to $x z < y z$. Thus, if $x < y$ and $z > 0$, then $x z < y z$.


	\item If $x<0$ then $x^{-1}<0$.
	      \\
	      \textbf{Proof.}\\
	      $\exists  x \in \mathbb{R}$ such that $x < 0$. Since $x < 0$, we have $-x > 0$. Since $-x > 0$, we can take the reciprocal of both sides to get $(-x)^{-1} > 0$. This implies $x^{-1} > 0$. Thus, if $x < 0$, then $x^{-1} > 0$.
	      $x < 0 \implies x-x < 0-x \implies -x > 0$.
	\item If $x, y \geq 0$ and $n \in \mathbb{Z}_{>0}$, prove that $x \leq y$ if and only if $x^n \leq y^n$. Deduce that $x<y$ if and only if $x^n<y^n$.
	      \\
	      \textbf{To Prove.} $x, y \geq 0$ and $n \in \mathbb{Z}_{>0}$, then $x \leq y \iff x^n \leq y^n$.
	      \underline{Case I}, $x \leq y$\\


\end{enumerate}









\newpage
\problem{C}{4 Points}
\begin{enumerate}
	\item Show that $\mathbb{Q}$ is neither bounded above nor bounded below.
	\item Determine the supremum and infimum of the following sets. Substantiate your claims with proofs: \\
	      \textbf{Definition.} Let $S \subseteq \mathbb{R}$ and $x \in \mathbb{R}$.\\
	      The element $x$ is the \emph{supremum} of $S$ if $s \leq x, \forall s \in S$ and for any upper bound $y$ of $S$, $x \leq y$. \\The element $x$ is the \emph{infimum} of $S$ if $s \geq x, \forall s \in S$ and for any lower bound $y$ of $S$, $x \geq y$.

	      \begin{enumerate}
		      \item (-1, 0] = \{$x \in \mathbb{R}, -1 < x \leq 0$\}\\
		            \textbf{To Prove.} Supremum is 0, Infimum is -1.\\
		            \textbf{Proof.} 0 is an upper bound by the definition of the set. Let $y$ be an upper bound for (-1, 0], since $0 \in (-1, 0] \implies 0 \leq y$. So, 0 is the supremum.\\
		            -1 is a lower bound by the definition of the set. To show that -1 is the greatest lower bound, lets assume $M$ is the infimum and $M > -1$ ($M$ cannot be less than -1 since -1 is already a lower bound, and any number lesser than -1 cannot be the infimum).\\
		            Let $x$ be the midpoint between $M$ and -1, $x = \frac{M + (-1)}{2} = \frac{M - 1}{2} = M - (\frac{1+M}{2}).$ This implies $M > x$, and we have to show that $x > -1$.\\
		            But let's assume $x \leq -1$, then
		            \begin{align*}
			            M - (\frac{1+M}{2}) & \leq -1 \\
			            M - 1               & \leq -2 \\
			            M                   & \leq -1
		            \end{align*}
		            But we assumed $M > -1$, thus by contradiction, $x > -1$. This implies $M > x > -1$, which contradicts the assumption that $M$ is the infimum. Thus, -1 is the infimum.

		      \item (1, 2) = \{$x \in \mathbb{R}, 1 < x < 2$\}\\
		            \textbf{To Prove.} Supremum is 2, Infimum is 1.\\
		            \textbf{Proof.} 2 is an upper bound by the definition of the set. To show that 2 is the least upper bound, let's assume $M$ is the supremum and $M < 2$ ($M$ cannot be greater than 2 since 2 is already an upper bound, and any number greater than 2 cannot be the supremum).\\
		            Let $x$ be the midpoint between $M$ and 2, $x = \frac{M + 2}{2}$. This implies $M < x$, and we have to show that $x < 2$.\\
		            But let's assume $x \geq 2$, then
		            \begin{align*}
			            \frac{M + 2}{2} & \geq 2 \\
			            M + 2           & \geq 4 \\
			            M               & \geq 2
		            \end{align*}
		            But we assumed $M < 2$, thus by contradiction, $x < 2$. This implies $2 > x > M$, which contradicts the assumption that $M$ is the supremum. Thus, 2 is the supremum.\\
		            $\square$\\


		            1 is a lower bound by the definition of the set. To show that 1 is the greatest lower bound, lets assume $M$ is the infimum and $M < 1$ ($M$ cannot be greater than 1 since 1 is already a lower bound, and any number greater than 1 cannot be the infimum).\\
		            Let $x$ be the midpoint between $M$ and 1, $x = \frac{M + 1}{2}$. This implies $M < x$, and we have to show that $x < 1$.\\
		            But let's assume $x \geq 1$, then
		            \begin{align*}
			            \frac{M + 1}{2} & \geq 1 \\
			            M + 1           & \geq 2 \\
			            M               & \geq 1
		            \end{align*}
		            But we assumed $M < 1$, thus by contradiction, $x < 1$. This implies $1 > x > M$, which contradicts the assumption that $M$ is the infimum. Thus, 1 is the infimum.
		            $\square$
	      \end{enumerate}
\end{enumerate}


















\newpage
\problem{D}{4 Points}
\begin{enumerate}
	\item Let $a, b \in \mathbb{R}$
	      \begin{enumerate}
		      \item Consider the set $\{a, b\}$. Prove that the supremum of the set $\{a, b\}$, is equal to $a$ if if $a \geq b$ and is equal to $b$ of $b \geq a$.
		            \\
		            \textbf{Solution.} \\
		            \underline{Case I}, Let $a, b \in \mathbb{R}$ and $a \geq b$.
		            \\
		            $\forall x \in \{a, b\}, x \leq a$ therefore $a$ is an upper bound.\\
		            Let $y$ be an upper bound for $\{a, b\}$, since $a \in \{a, b\} \implies a \leq y$. So, $a$ is the supremum.\\
		            \\
		            \underline{Case II}, Let $a, b \in \mathbb{R}$ and $b \geq a$.
		            \\
		            $\forall x \in \{a, b\}, x \leq b$ therefore $b$ is an upper bound.\\
		            Let $y$ be an upper bound for $\{a, b\}$, since $b \in \{a, b\} \implies b \leq y$. So, $b$ is the supremum.\\

		      \item \textbf{To Prove.}  $$
			            \sup (\{a, b\})=\frac{a+b+|a-b|}{2} .
		            $$
		            \textbf{Solution.} \\
		            \underline{Case I}, $a \geq b$\\
		            From QD.1(a), we know that the supremum of the set $\{a, b\}$ is $a$
		            \begin{align*}
			            \text{sup}(\{a, b\}) & = a                         \\
			            \text{sup}(\{a, b\}) & = a + b - b                 \\
			            \text{sup}(\{a, b\}) & = \frac{2a +b - b}{2}       \\ \text{Since } a \geq b, a - b = |a - b| \\
			            \text{sup}(\{a, b\}) & = \frac{a + b + |a - b|}{2}
			            \\ \square
		            \end{align*}
		            \\
		            \underline{Case II}, $b \geq a$\\
		            From QD.1(a), we know that the supremum of the set $\{a, b\}$ is $b$
		            \begin{align*}
			            \text{sup}(\{a, b\})                 & = b                     \\
			            \text{sup}(\{a, b\})                 & = a + b - a             \\
			            \text{sup}(\{a, b\})                 & = \frac{a+b-(a-b)}{2}   \\
			            \text{Since } a \leq b, a - b \leq 0 & \implies |a-b| = -(a-b) \\
			            \text{sup}(\{a, b\})                 & = \frac{a+b+|a-b|}{2}
			            \\ \square
		            \end{align*}
		            Thus for both cases, the formula $\sup (\{a, b\})=\frac{a+b+|a-b|}{2} $ works. Hence proved.



		            \newpage
		      \item Formulate and prove a variant of the above formula describing the infimum of the set $\{a, b\}$.
		            \textbf{Solution.}\\
		            \underline{Case I}, $a \geq b$ \\
		            Using the same principle from QD.1(a)
		            \begin{align*}
			            \text{inf}(\{a, b\})          & = b                   \\
			            \text{inf}(\{a, b\})          & = b + a - a           \\
			            \text{inf}(\{a, b\})          & = \frac{b + b+a-a}{2} \\
			            \text{inf}(\{a, b\})          & = \frac{a+b-a+b}{2}   \\
			            \text{inf}(\{a, b\})          & = \frac{a+b-(a-b)}{2} \\
			            \text{Since } a \geq b, a - b & = |a-b|               \\
			            \text{inf}(\{a, b\})          & = \frac{a+b+|a-b|}{2}
			            \\ \square
		            \end{align*}
		            \underline{Case II}, $b \geq a$ \\
		            Using the same principle from QD.1(a)
		            \begin{align*}
			            \text{inf}(\{a, a\})          & = a                   \\
			            \text{inf}(\{a, b\})          & = a + b - b           \\
			            \text{inf}(\{a, b\})          & = \frac{a+b-b+a}{2}   \\
			            \text{inf}(\{a, b\})          & = \frac{b+b-(b-b)}{2} \\
			            \text{Since } a \geq b, a - b & = |a-b|               \\
			            \text{inf}(\{a, b\})          & = \frac{a+b+|a-b|}{2}
			            \\ \square
		            \end{align*}
	      \end{enumerate}
	\item Let $a_1, \cdots, a_n$ be elements of $\mathbb{R}$. Prove that $$
		      \sup \left(\left\{a_1, \ldots, a_n\right\}\right)=\sup \left(\left\{\sup \left(\left\{a_1, \ldots, a_{n-1}\right\}\right), a_n\right\}\right) .
		      \\
	      $$


	      Deduce that $\sup \left(\left\{a_1, \ldots, a_n\right\}\right) \in\left\{a_1, \ldots, a_n\right\}$.
	      \\
	      \textbf{Solution.}\\
	      \underline{Case I}, Let $a_1, \cdots, a_n$ be elements of $\mathbb{R}$ and $a_1 \geq a_2 \geq \cdots \geq a_n$.

\end{enumerate}







\newpage
\problem{E}{3 Points}
Let $S$ be a subset of $\mathbb{R}$.
\begin{enumerate}
	\item     Show that $S$ is bounded if and only if there exists an $m \in \mathbb{R}$ such that $|x| \leq m$ for all $x \in S$.
	      \\
	      \textbf{To Prove.} $S$ is bounded $\iff$ $\exists m \in \mathbb{R}$ such that $|x| \leq \forall x \in S$.
	      \\
	      \textbf{Proof.} Let $S \subseteq \mathbb{R}$ be a bounded set
	      From $E(1), \exists m \in \mathbb{R}$ such that $|x| \leq m \forall x \in S$
	      $$
		      \begin{aligned}
			       & \forall x \in S, x \leqslant m \text { and }-x \leqslant m \\
			       & \forall x \in S,-m \leqslant x \text { and } x \leqslant m \\
			       & \forall x \in S,-m \leqslant x \leqslant m                 \\
			       & \forall x \in S, S \in[-m, m]                              \\
			       & S \leq[-m, m] \quad[a \neq b]
		      \end{aligned}
	      $$

	      Assume $\exists m \in \mathbb{R}$ sit $S \subseteq[-m, m]$
	      $$
		      \Rightarrow \forall x \in S,-m \leqslant x \leqslant m
	      $$
	      $\Rightarrow \forall x \in S,-m \leqslant x \Rightarrow-m$ is a lower bound
	      $\Rightarrow \forall \in S, x \leqslant m \Rightarrow m$ is an upper bound of $S$.
	      Since $S$ has a lower bound and an upper bound, $S$ is bounded in $\mathbb{R}$.
	      $$
		      [b \Rightarrow a]
	      $$	\item  Deduce that the following three statements are equivalent:
	      \begin{enumerate}
		      \item The $S$ is bounded.
		      \item There exists $m \in \mathbb{R}$ such that $S \subseteq[-m, m]$.
		      \item There exist $a, b \in \mathbb{R}$ with $a \leq b$ such that $S \subseteq[a, b]$.
	      \end{enumerate}
\end{enumerate}









\newpage
\problem{F}{4 Points}
\begin{enumerate}
	\item Let $x, y \in \mathbb{R}$, and suppose $x$ and $y$ both satisfy the definition of an infimum of $S$. Show that $x=y$.
	      \textbf{To Prove.} If $x, y \in \mathbb{R}$ and both satisfy the definition of an infimum of $S$, then $x=y$.\\
	      \textbf{Definition.} Let $S \subseteq \mathbb{R}$. We say that $x \in \mathbb{R}$ is the infimum of $S$ if $x \leq s  \forall s \in S$ and for any lower bound $y$ of $S$, $x \geq y$.\\
	      \textbf{Proof.} Suppose $x$ and $y$ are the infimum of $S$.
	      \\ Since $y$ is an infimum of $S$, $y$ is a lower bound of $S$. Therefore, by the property of infimum, $y\leq x$ because $x$ is the infimum.
	      \\ Since $x$ is an infimum of $S$, $x$ is a lower bound of $S$. Therefore, by the property of infimum, $x \geq y$ because $y$ is the infimum.
	      \\ By the property of antisymmetry, $x \geq y$ and $y \geq x$ implies $x = y$. Thus, $x = y.
		      \\\square$


	\item Let $x \in \mathbb{R}$. Show that $x=\inf (S)$ if and only if it satisfies the following two conditions:

	      \begin{enumerate}
		      \item The element $x$ is a lower bound for $S$.
		      \item Given $\epsilon>0$, there exists some $y \in S$ such that $y<x+\epsilon$.
		            \\\textbf{To Prove.} $x=\inf (S) \iff  x \text{ is a lower bound for } S \text{ and } \forall \epsilon > 0, \exists y \in S \text{ such that } y < x + \epsilon$.
		            \\
		            \textbf{Solution.} \\
		            (a) It's already in the definition of infimum that $x$ has to be a lower bound, so no need to prove that.
		            \\
		            (b) If such $y$ does not exist, $\forall y \in S, y \geq x + \epsilon \implies x + \epsilon$ is a lower bound of $S$. $x+\epsilon$ is a lower bound > $x \implies x$ is not infimum.
		            \\
		            If such $y$ exists, then $x + \epsilon$ is not a lower bound of $S$. Thus, $x$ is the infimum of $S$.


	      \end{enumerate}
	\item Let $x \in \mathbb{R}$. Show that $x=\inf (S)$ if and only if it satisfies the following two conditions:
	      \begin{enumerate}
		      \item The element $x$ is a lower bound for $S$.
		      \item Given $n \in \mathbb{Z}_{>0}$, there exists some $y \in S$ such that $y<x+\frac{1}{n}$.
	      \end{enumerate}
\end{enumerate}


\newpage
\problem{G}{4 Points}
Prove that in the presence of properties (1) to (16), property (17) and property $(17)_L$ are equivalent. Deduce that every non-empty bounded below subset of $\mathbb{R}$ has an infimum.
\\

\noindent \textbf{To Prove.} If $A$ and $B$ are subsets of $\mathbb{R}$ such that both $A$ and $B$ are non-empty and $a \leq b$ for all $a \in A$ and $b \in B$, there exists $s \in \mathbb{R}$ such that $a \leq s \leq b$ for all $a \in A$ and $b \in B \iff$ every non-empty bounded below subset of $\mathbb{R}$ has an infimum.
\\
\\
\textbf{Proof.}\\
\underline{Assuming 17 $\implies$ $17_L$}\\
Property $(17)_L$ states that every non-empty bounded below subset of $\mathbb{R}$ has an infimum. Let $S \subseteq \mathbb{R}$ be a non-empty bounded below subset. Define $B:=\{x \in \mathbb{R} \mid x \leq s$ for all $s \in S\}$, the set of all lower bounds of $S$. Since $S$ is bounded below, $B$ is non-empty.

\noindent By property (17), there exists $t \in \mathbb{R}$ such that $t \leq s$ for all $s \in S$ and $b \leq t$ for all $b \in B$. Thus, $t$ is a lower bound for $S$ (since $t \leq s$ for all $s \in S$ ) and $t$ is greater than or equal to any other lower bound of $S$ (since $b \leq t$ for all $b \in B$ ). Therefore, $t$ is the infimum of $S$.

\noindent We conclude that property $(17)_L$ holds.
\\
\\
\noindent \underline{Assuming $17_L$ $\implies$ 17}\\
Assume property $(17)_L$ holds. We need to show that property (17) also holds.

\noindent Property (17) states that given non-empty subsets $A$ and $B$ of $\mathbb{R}$ such that $a \leq b$ for all $a \in A$ and $b \in B$, there exists $t \in \mathbb{R}$ such that $a \leq t \leq b$ for all $a \in A$ and $b \in B$.

\noindent Let $A$ and $B$ be non-empty subsets of $\mathbb{R}$ such that $a \leq b$ for all $a \in A$ and $b \in B$. This condition implies that every element of $B$ is a lower bound for $A$, so $A$ is bounded below. By property $(17)_L$, $A$ has an infimum, say $t$. Since $t$ is a lower bound for $A$, we have $t \leq a$ for all $a \in A$. Moreover, since $t$ is the greatest lower bound of $A$, and every element of $B$ is a lower bound for $A$, we have $b \leq t$ for all $b \in B$.

\noindent Thus, there exists $t \in \mathbb{R}$ such that $a \leq t \leq b$ for all $a \in A$ and $b \in B$.

\noindent We conclude that property (17) holds.
~\\
~\\
\noindent Therefore 17 $\iff 17_L$. \noindent $\square$


\end{document}
